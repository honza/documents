\documentclass[letterpaper]{report}
%\usepackage[pass]{geometry}
\usepackage{fontspec}
\setmainfont [Ligatures={Common,TeX}, Numbers={OldStyle}]{Baskerville}

\begin{document}

\title{Advantages of Sphinx over Google Sites}
\author{Honza Pokorny}
\date{November 2011}
\maketitle

\section*{Introduction}

    It is my pleasure to present this report to SheepDogInc's management. In
    this report, I will argue that Google Sites is an inefficient tool for
    writing documentation for teams of developers and will suggest that we move
    to use a tool called Sphinx. This report assumes a general understanding of
    and familiarity with Google Sites.

\section*{Problems with Google Sites}

    Let us first outline some of the major issues that arise when working with
    Google Sites to produce product documentation.

    \subsection*{Proprietary technology}

        Google Sites is a hosted solution and does not provide the developer
        with its source code. Therefore, the developer cannot be certain that
        the solution will exist in the future. By continuing to use Google
        Sites, SheepDogInc is trusting that Google will keep the product alive
        indefinitely. If Google Sites were to be phased out today, SheepDogInc
        developers would be force spend considerable amount of time porting the
        content to a different solution.

    \subsection*{Limited customizability}

        Google Sites is targeted at small businesses and users with minimal
        computer skills. As a result, the customization options available are
        very limited. Custom CSS and Javascript cannot be incorporated into a theme
        template\footnote{http://en.wikipedia.org/wiki/Google\_Sites\#Limitations}.
        If SheepDogInc is interested in having full control over their product
        branding, the use of Google Sites should be discouraged.

    \subsection*{Professionalism}

        SheepDogInc is a professional web development company. As such, hosting
        the documentation for a SheepDogInc product on Google Sites may appear
        unprofessional to Internet-savvy users and potential customers.

\section*{Sphinx primer}

    \subsection*{Introduction}

        Sphinx is a widely used tool that makes it easy to create intelligent
        and beautiful documentation. It was originally created for the Python
        programming language documentation.

        Sphinx documentation is written in plain text files that are easy to
        read. Once the documentation is ready for publishing, Sphinx compiles
        it to a number of formats such as HTML and PDF. Any customizations to
        the presentation of the final document are done independently of the
        content by creating a custom Sphinx theme. Sphinx themes are
        collections of CSS, Javascript and images.

    \subsection*{Adoption}

        Sphinx is used by hundreds of open source projects, including the
        Python programming language and  Django. It is the industry standard
        when it comes to documenting Python projects. As you can see, Sphinx is
        widely used and is in constant development.

    \subsection*{License}

        Sphinx is an open source project licensed under the terms of the BSD
        license. This allows the developer to use, study, modify and distribute
        the software free of charge.

\section*{Key benefits}

    \subsection*{Version Control}

        The source files for Sphinx-produced documentation can be
        version-controlled using git like any other source code and kept in the
        same repository as the product or project. The developer does not need
        to remember how to look up the appropriate Google Site; he can just
        open the source file on his computer.

        Moreover, when a developer checks in a new feature he or she can submit
        a few paragraphs of documentation in the same pull request. This is
        common practice in large open source projects where a patch will not be
        accepted without
        documentation\footnote{https://docs.djangoproject.com/en/1.3/internals/contributing/\#patch-style}.

        All benefits of version control apply here. Multiple developers can
        easily collaborate on the same piece of documentation. New releases of
        the documentation can be tagged. Every single change to the
        documentation is recorded and can be reverted later.

    \subsection*{Plain text files}

        In general, programmers prefer text files over proprietary document
        formats such as Microsoft Word. The developer can be certain that he
        will always be able to open a text file. Developers spend most of their
        time editing text files. Their text editors or IDEs are heavily
        customized to make editing of text as efficient as possible.

    \subsection*{Style and presentation}

        The problem with WYSIWYG\footnote{http://en.wikipedia.org/wiki/WYSIWYG}
        editors is that one can easily create graphically inconsistent
        documents. Heading sizes and levels may not match. The color of warning
        text may be of two different shades of red. Spacing between paragraphs
        may differ.

        With Sphinx, everything is consistent. The result is a clean,
        consistent HTML document that appears polished and professional.

    \subsection*{PDF generation}

        Some Internet users prefer to print large documents off because long
        periods of reading on an LCD screen hurts their eyes. Sphinx creates a
        professional quality PDF file without any changes to the source files.
        This generated PDF file can be made available for download on the
        documentation page.

    \subsection*{Quality Assurance}

        Any changes to the documentation will pass through internal processes
        like any other source code. A pull request will be opened and someone
        will perform a code review. Before the changes can be released, the
        document will be checked for quality.

\section*{Potential drawbacks}

    \subsection*{Learning curve}

        Each developer will be required to learn the basics of the system. They
        will need to learn how to create headings, paragraphs, lists, links and
        style basic texts. This is estimated to take less than ten minutes of
        each person's time.

        The actual conversion of the document will be highly automated as part
        of the continuous deployment process already in place for the given
        project.

    \subsection*{Access}

        Given the nature of the software that produces the HTML output, it may
        be difficult for a non-developer in the organization to make simple
        changes. For instance, project managers may be required to contact a
        developer to correct a misspelling found in the text.

\section*{Summary}

    In conclusion, Google Sites as a documentation tool is unsuitable for high
    quality Web products created by SheepDogInc. By adopting Sphinx as the main
    tool for writing documentation, SheepDogInc has the potential to be
    successful in the marketplace while remaining efficient and productive.

\end{document}
